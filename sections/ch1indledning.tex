\chapter{Indledning}
\section{Kernebegreber}
\subsection{Cloud computing}
Som det står skrevet i NIST’s definition af cloud computing: “Cloud computing is a model for enabling ubiquitous, convenient, on-demand network access to a shared pool of configurable computing resources (e.g., networks, servers, storage, applications, and services) that can be rapidly provisioned and released with minimal management effort or service provider interaction. This cloud model is composed of five essential characteristics, three service models, and four deployment models. \citep{cloud_def}” Så er cloud  computing en fælles sværm af computer ressourcer som kan tilgås fra hvor som helst.\\

\subsection{Små eller mellemstore virksomheder(SMV)}
Små eller mellemstore virksomheder forkortet SMV står for størrelsen af virksomheden. Navnet bliver defineredet af 2 faktorer; antal af medarbejdere og omsætning. Små virksomheder har mellem 10 og 50 medarbejdere samt en omsætning på mindre 10 mio. euro. hvor en mellemstor virksomhed har mellem 50 og 250 medarbejdere og en omsætning på mindre end 43 mio. euro. hvis en virksomhed fx har 30 medarbejdere men har en omsætning på mere end 10 mio. euro, bliver størrelsen af firmaet defineret af højeste nævner. \citep{smvdef}.

\subsection{Uptime & Downtime}
Uptime er den periode hvor en service er kan bruges. Normalt regnes  Downtime er en tidsperiode hvor en service eller et produkt ikke kan benyttes. Ofte forbundet med software, og specielt cloud services. Sker som regel i forbindelse med vedligeholdelse af service-holderens servere. 