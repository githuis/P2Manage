\chapter{Indledning}
Nye og mindre virksomheder har i dag svært ved at udvide, da en udvidelse kræver større og mere omfattende administration \citep{SmaaFirmaerOrker}. Da virksomheder ofte er startet op af specialister indenfor virksomhedens felt og ikke indenfor administration, kan en sådan udvidelse skræmme \citep{SmaaFirmaerOrker}. Der findes dog adskillige it-værktøjer, der kan afhjælpe netop dette problem. Disse værktøjer har dog hver især deres fejl og mangler, og det faktum at de er dyrere end at bruge manuelle regneark som f.eks. Excel \citep{Play}\ref{app:subway}.\\
Dårligt arbejdsmiljø er et andet betydeligt problem, ikke kun for små virksomheder, men også større virksomheder. En IT-løsning indenfor administration, ville her kunne afhjælpe problemet, ved f.eks. at inddrage medarbejdere i vagtplanlægingen, eller give dem mulighed for f.eks. at tjekke op på deres egen løn. Samtidig ville det give chefen mere tid til andet arbejde, samt sikre, en større overskuelighed over medarbejderne og virksomheden. 
Arbejdsmiljøet har en effekt på produktiviteten hos medarbejderne, hvilket vil sige, at glade medarbejdere yder både mere og giver bedre service \citep{Jensen2014}. Derfor vil denne rapport tage udgangspunkt i den initierende problemstilling \textit{Hvordan påvirker IT administration arbejdsmiljøet i en virksomhed}.\todo{Vi skal have lavet nogle underspørgsmål som leder over i vores problemanalyse} 

\section{Kernebegreber}
For få den fulde forståelse af rapporten, er der en række termer der er anbefalet at sætte sig ind i. Disse termer er defineret nedenfor og dækker over den forståelse som de bliver andvendt med i rapporten. 
\subsection{Cloud computing}
Som det står skrevet i  National Institute
of Standards and Technology’s definition af cloud computing: \textit{"Cloud computing is a model for enabling ubiquitous, convenient, on-demand network access to a shared pool of configurable computing resources"} \citep{cloud_def}. Så cloud computing er en fælles sværm af computer ressourcer og services som kan tilgås fra hvor som helst.

\subsection{Små eller mellemstore virksomheder}
Små eller mellemstore virksomheder, forkortet SMV, står for størrelsen af virksomheden. Navnet bliver defineredet af to faktorer; antal af medarbejdere og omsætning. Små virksomheder har mellem 10 og 50 medarbejdere samt en omsætning på mindre \euro 10 mio.. hvor en mellemstor virksomhed har mellem 50 og 250 medarbejdere og en omsætning på mindre end \euro 43 mio. Hvis en virksomhed f.eks. har 30 medarbejdere, men har en omsætning på mere end \euro 10 mio., bliver størrelsen af firmaet defineret af højeste nævner, i dette tilfælde som en mellemstor virksomhed \citep{SMV}. Yderligere anses afdelinger af samme størrese også for at være SMV, i dette projekt. Dette gøres for at give løsningen så stor en målgruppe som muligt.

\subsection{Uptime \& Downtime}
Uptime er den periode hvor en service eller et produkt kan bruges. Normalt regnes en service for brugbar hele tiden, men dette er ikke altid tilfældet. Downtime er tidsperioden hvor en service eller et produkt ikke kan benyttes. Begreberne Uptime og Downtime er forbundet med software, og specielt cloud services \citep{drpbx_downtime, UpDown}. Downtime sker som regel i forbindelse med vedligeholdelse af udbyderens servere \citep{drpbx_downtime}.