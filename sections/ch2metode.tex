\chapter{Metodeafsnit}
\section{AAU modellen}
Aalborgmodellen for Problembaseret Læring, forkortet PBL, er en model for formidling og indlæring, beregnet til studerende. Aalborgmodellen tager udgangspunkt i gruppebaseret læring og arbejde. Udgangspunktet for læringsprocessen er et initierende problem, hvor viden tilegnes gennem projektarbejde, ud fra et bredt teoretisk perspektiv. De studerende styrer selv projektet og skal selv udarbejde projektet. Gruppen har adgang til projektvejledning og med den gensidige kritik opnås de bedste resultater. Der bliver lagt tryk på samarbejde, feedback og refleksion som de studerende tilegner sig via PBL-Aalborgmodellen.
Aalborgmodellen bruges til at tilegne en række kompetencer til dem der bruger den. Fra Aalborg universitets hjemmeside er der fundet følgende eksempler på kompetencer:
\begin{itemize}
\item {Tilegne sig viden og færdigheder selvstændigt og på et højt fagligt niveau.}
\item {Arbejde analytisk, tværfagligt og problem- og resultatorienteret.}
\item {Samarbejde med erhvervslivet om løsning af autentiske faglige problemer.}
\item {Udvikle evner inden for teamwork.}
\item {Blive forberedt til arbejdsmarkedet.}
\end{itemize}
\citep{Universitet2015}\citep{Universitet2011}

\section{Kvalitativ og kvantiativ metode}
\textbf{Kvalitativ metode:}
De kvalitative metoder bruges f.eks. til at analysere og fortolke tekst såvel som andet materiale. I stedet for f.eks. at se på hvor mange gange en politiker bruger et bestemt ord i sin tale, så undersøges betydningen bag hvert enkelt ord og den sammenhæng ordet skal forstås i \citep{Gymportalen}. De kvalitative metoder bliver brugt når dataen er svær at kvantificere og der er tale om forskningsfeltet mere som et subjekt, frem for et objekt. Dataen er derfor mere nuanceret og bliver undersøgt ved interaktion mellem subjektet og forskeren \citep{Kval}.\\ 

\noindent\textbf{Kvantitativ metode:}
I de kvantitative metoder indsamlere man en større mængde data, oplysningerne kan oftest kvantificeres og ses oftes som statistikker. Dette kan f.eks. være spørgeskemaundersøgelser, hvor man spørger en stor gruppe mennesker med en række simple spørgsmål, for at påvise en sammenhæng til virkeligheden. Personudvælgelsen til undersøgelsen er tilfældig, medmindre undersøgelsen er tilrettet en bestemt målgruppe \citep{Kvan}. I spørgeskemaundersøgelser stiller man som regel en række konkrete lukkede spørgsmål, der kan svares med et ja eller nej. Svarene bliver herefter behandlet statistisk, så man bagefter kan måle oplysningerne. Andres statistikker, som allerede er lavet, bliver benyttet til at uddrage information fra, da omfanget af oplsyningerne der er målt, ikke vil kunne genberegnes af gruppen selv \citep{Gymportalen}.\\ 

\noindent I dette projekt er der både anvendt kvalitative metoder samt kvantitative. De kvantitaive metoder er anvendt i form af statestikker, til at understøtte teser samt uddrage information. De kvalitative metoder er anvendt i kvalitative interviews. Disse bliver brugt til at uddrage information om et emne.

\section{Brug af kilder og kildekritik}
I projektet gøres der brug af skrevne kilder som bøger, artikler og hjemmesider. Dette gøres da der allerede er foretaget relevante undersøgelser omkring emnet på tidligere tidspunkter. Dog er det ikke alt information der kan anvendes, derfor bliver hvert enkelt dokument analyseret, med henblik på at sikre kildens troværdighed. Desuden er det ofte kun nyere artikler som kan bruges i projektet og derfor undersøges kildens samtidighed først, for at sikre at vigtige ændringer ikke er sket efter kildens oprettelse. Derefter undersøges der hvem forfatteren er, hvilket fagligt forhold forfatteren har til emnet og om forfatteren holder sig objektivt eller subjektivt til emnet. Hvis alt dette stemmer overens, accepteres kilden og den information som kilden indeholder vil derefter blive analyseret og anvendt i projektet. På denne måde er det muligt at opbygge en rapport med flere undersøgelser og mere viden, end det som projektdeltagerne selv er i stand til at bidrage med \citep{Kildekritik}.

\section{Interessentanalyse}
Interessentanalysen benyttes i dette projekt til at identificere de vigtigste interessenter. Ved at anvende denne analyse, findes der frem til hvem og hvad, der har indflydelse på problemet. Når interessenterne er fundet, bruges interessentanalysen til at kategorisere dem. Derudover bruges interessentanalysen til at undersøge interessenternes ønsker, samt hvilket udbytte og hvilke fordele projektet har for dem. Til sidst bliver der undersøgt hvad man kan forvente, at de vil bidrage med, positivt og negativt \citep{MetteLindegaardAttrup2008}.

\section{Interview}
Et interview bruges til at samle kvalitativ information fra én bestemt person eller en mindre gruppe mennesker. I projektet benyttes interviews til at søge information fra personer, som har erfaring med eller forståelse for emnet, såsom eksperter eller nøglepersoner. De interviews der bruges i dette projekt, bruges til at give indsigt og viden indenfor emnet og give information, som kan bruges senere i projektet. De vil give information som kan bruges til indsnævring af en problemstilling og til udvikling af mulige løsningsforslag.
Indenfor interviews findes der flere forskellige måder at håndtere dette på. I rapporten bliver det kvalitative forskningsinterview anvendt, da man ved hjælp af denne metode kan undersøge de enkelte meninger og udforske holdninger. Denne metode er god til at få uddybende svar, men derimod knapt så god til at kvantificere.
Det kvalitative forskningsinterview er et semistruktureret interview i denne opgave, da denne metode giver mulighed for uddybende svar omkring emnet, men giver samtidig mulighed for at komme ind på områder, som måske ikke er blevet tænkt over inden. Så på den måde kommer informanternes svar til at påvirke projektet \citep{BjarneHjorthAndersen, kvale2009}.

\section{Teknologivurdering}
Ved en teknologivurdering undersøges forskellige eksisterende systemer. I dette projekt undersøges alternative arbejdsplanlægnings værktøjer og deres funktionaliteter. Ved hjælp af denne undersøgelse, vil der blive mulighed for at se, hvad nuværende systemer er i stand til rent funktionsmæssigt. Dermed identificeres nuværende problemstillinger ved systemerne, der kan tages forbehold for i program udviklingsprocesen, og dermed optimere det endelige produkt \citep{PeterLarsen}.

%https://www.moodle.aau.dk/pluginfile.php/394871/mod_resource/content/1/Teknologivurdering%20SW%20DAT%202014.pdf


