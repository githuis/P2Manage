\documentclass[11pt, twoside, a4paper, openright]{report}

\usepackage[utf8]{inputenc}
\usepackage[danish]{babel}
\usepackage{graphicx}
\usepackage{listings}
\usepackage{array, booktabs}
\usepackage[inner=28mm, outer=41mm]{geometry}
\usepackage{fancyhdr}
\usepackage{calc}

\usepackage{hyperref}
\hypersetup{% 
pdfpagelabels=true,%
plainpages=false,%
pdfauthor={Author(s)},%
pdftitle={Title},%
pdfsubject={Subject},%
bookmarksnumbered=true,%
colorlinks,% 
citecolor=black,%
filecolor=black,%
linkcolor=black,%
urlcolor=black,% 
pdfstartview=FitH%
}

\begin{document}

\input{Input/frontpage}
\input{Input/colophon}
\newcommand{\aautitlepage}[3]{%
  {
    %set up various length
    \ifx\titlepageleftcolumnwidth\undefined
      \newlength{\titlepageleftcolumnwidth}
      \newlength{\titlepagerightcolumnwidth}
    \fi
    \setlength{\titlepageleftcolumnwidth}{0.5\textwidth-\tabcolsep}
    \setlength{\titlepagerightcolumnwidth}{\textwidth-2\tabcolsep-\titlepageleftcolumnwidth}
    %create title page
    \thispagestyle{empty}
    \noindent%
    \begin{tabular}{@{}ll@{}}
      \parbox{\titlepageleftcolumnwidth}{
        \iflanguage{danish}{%
          \includegraphics[width=\titlepageleftcolumnwidth]{figures/aau_logo_da}
        }{%
          \includegraphics[width=\titlepageleftcolumnwidth]{figures/aau_logo_en}
        }
      } &
      \parbox{\titlepagerightcolumnwidth}{\raggedleft\sf\small
        #2
      }\bigskip\\
       #1 &
      \parbox[t]{\titlepagerightcolumnwidth}{%
      \textbf{Abstract:}\bigskip\par
        \fbox{\parbox{\titlepagerightcolumnwidth-2\fboxsep-2\fboxrule}{%
          #3
        }}
      }\\
    \end{tabular}
    \vfill
    \iflanguage{danish}{%
      \noindent{\footnotesize\emph{Rapportens indhold er frit tilgængeligt, men offentliggørelse (med kildeangivelse) må kun ske efter aftale med forfatterne.}}
    }{%
      \noindent{\footnotesize\emph{The content of this report is freely available, but publication (with reference) may only be pursued due to agreement with the author.}}
    }
    \clearpage
  }
}

%Create danish project info
\newcommand{\danishprojectinfo}[8]{%
  \parbox[t]{\titlepageleftcolumnwidth}{
    \textbf{Titel:}\\ #1\bigskip\par
    \textbf{Tema:}\\ #2\bigskip\par
    \textbf{Projektperiode:}\\ #3\bigskip\par
    \textbf{Projektgruppe:}\\ #4\bigskip\par
    \textbf{Deltager(e):}\\ #5\bigskip\par
    \textbf{Vejleder(e):}\\ #6\bigskip\par
    \textbf{Oplagstal:} #7\bigskip\par
    \textbf{Sidetal:} \pageref{LastPage}\bigskip\par
    \textbf{Afleveringsdato:}\\ #8
  }
}
{\selectlanguage{danish}
\pdfbookmark[0]{Danish title page}{label:titlepage_da}
\aautitlepage{%
  \danishprojectinfo{
    Rapportens titel %title
  }{%
    Programmering og Problemløsning %theme
  }{%
    Forårssemestret 2015 %project period
  }{%
    A423 % project group
  }{%
    %list of group members
    Andreas Smed Lauritsen\\ 
    Daniel Dirk Albert van Bolhuis\\
    Jesper Hedegaard\\
    Luca Bianchi Andersen\\
    Jens Birkbak Aagaard\\
    Ronni Lenvig Hansen
  }{%
    %list of supervisors
    Jane Billestrup
  }{%
    0 % number of printed copies
  }{%
    \today % date of completion
  }%
}{%department and address
  \textbf{Datalogi}\\
  Aalborg Universitet\\
  \href{http://www.aau.dk}{http://www.aau.dk}
}{% the abstract
  Her er resuméet ikke endnu
}}


\tableofcontents

\chapter{Indledning}
	\section{Initierende problem}
	
\chapter{Metodeafsnit}
	\section{AAU Modellen}
	AAU modellen for Problembaseret Læring, forkortet PBL, er en anerkendt model for formidling og indlæring, beregnet til studerende. AAU modellen tager udgangspunkt i gruppebaseret læring og arbejde. Udgangspunktet for læringsprocessen er et initialiserende problem, hvor viden tilegnes gennem projektarbejde, ud fra et bredt teoretisk perspektiv. De studerende styrer selv projektet og skal selv udarbejde projektet. Gruppen har adgang til projektvejledning og med den gensidige kritik opnås de bedste resultater. Der bliver lagt tryk på samarbejde, feedback og refleksion som de studerende tilegner sig via PBL-Aalborgmodellen.
\\\\
Fordele med AAU modellen: 
\begin{itemize}
\item Tilegne sig viden og færdigheder selvstændigt og på et højt fagligt niveau
\item Arbejde analytisk, tværfagligt. problem- og resultatorienteret
\item Samarbejde med erhvervslivet om løsning af autentiske faglige problemer
\item Udvikle deres evner inden for teamwork
\item Blive godt klædt på til arbejdsmarkedet
\end{itemize}

\pagebreak
	\section{Kvalitativ Kvantitativ}
	Kvalitativ metode:
	 
Den kvalitative metode bruges til at analysere og fortolke tekst såvel som andet materiale. I stedet for fx at se på hvor mange gange en politiker bruger et bestemt ord i sin tale, så undersøges betydningen bag hvert enkelt sætning og den sammenhæng sætningen skal forståes i. Metoden går dermed ud på at se sammenhæng og subjektive detaljer i mindre materiale omfang.
\\\\
	Kvantitativ metode:

I den kvantitative metode foretages undersøgelser. Dette kan fx være spørgeskemaundersøgelser, hvor man spørger en stor gruppe mennesker en række relativt simple spørgsmål, for at påvise en sammenhæng til virkeligheden. Personudvælgelsen til undersøgelsen kan være tilfældig, medmindre ens undersøgelse er tilrettet en bestemt målgruppe. I spørgeskemaundersøgelser stiller man som regel en række konkrete lukkede spørgsmål, der kan svares med ja eller nej. Åbne spørgsmål, der svares i sætninger, giver mere interessante svar, men det tager længere tid at analysere når man har men en større gruppe at gøre. Denne metode benytter sig derfor at større materiale omfang men forholder sig til problemerne meget objektivt.
\\\\
	Rapportens brug af den kvalitative metode:

I dette projekt anvendes den kvalitative metode til at analysere tekster og  interviews. Den bruges til at analysere og fortolke den information som dokumenterne, relateret til emnet, indeholder. Den kvalitative metode er blevet valgt frem for den kvantitative metode, idet at den med henblik på projektet fremstår som det mest hensigtsmæssige valg. Dette er blandt andet fordi projekts problem kun fremstå hos en begrænset gruppe og det er derfor hensigtsmæssigt at analysere deres individuelle subjektive mening om problemet frem for at foretage en objektiv analyse.

		\subsection{Interview}
	Et interview bruges til at samle kvalitativ information fra én bestemt person eller en mindre gruppe mennesker. I projektet benyttes interview til at søge information fra personer, som har erfaring med eller forståelse for emnet, såsom eksperter eller nøglepersoner. Disse interviews vil bruges til at få indsigt og viden indenfor emnet og give information, som kan bruges senere i projektet. De vil give information som kan bruges til indsnævring af en problemstilling og til udvikling af mulige løsningsforslag.
	
	\section{Dokumentanalyse}
	
	\section{Interessentanalyse}
	Interessentanalysen benyttes i dette projekt til at identificere de vigtigste interessenter. Ved at anvende denne analyse, findes der frem til interessenterne, som har indflydelse på projektet. Når interessenterne er fundet, bruges interessentanalysen til at kategorisere interessenterne op i fire kategorier, som vist i figur \ref{fig:interesser} 
	\\\\
	\begin{figure}[h]
\centering
\includegraphics[scale=0.5]{../../Pictures/P2/Interessentanalyse.PNG}
\caption{De fire interesse områder}\label{fig:interesser}
	\end{figure}
	
	Derudover bruges interessentanalysen til at kigge på interessenternes ønsker, hvilket udbytte og hvilke fordele projektet har for dem. Til sidst bliver der undersøgt hvad man kan forvente, at de vil bidrage med, positivt og negativt.
	
	\section{Teknologivurdering}
	Ved en teknologivurdering bliver der testet forskellige nuværende systemer, der hjælper folk med at lægge vagtplaner. Ved hjælp af denne vurdering, vil der blive mulighed for at se, hvad nuværende systemer er i stand til. Dermed identificeres nuværende problemstillinger ved systemerne, der kan tages forbehold for ved program udviklingsproces, og dermed optimere det endelige produkt.
	
	\section{Kildekritik}
	Brugen af kilder er brugbar for dette projekt, fordi der allerede er foretaget undersøgelser omkring emnet på tidligere tidspunkter. Derfor kan brugbar information findes ved at gennemgå disse undersøgelser og lede efter information, som er relevant for projektet. Hvis relevant information findes vil denne information så blive analyseret og kildekritisk bedømt før det anvendes i projektet. Informationens og kildens troværdighed vil blive undersøgt, herunder hvem forfatteren er, og om forfatteren forholder sig objektivt eller subjektivt. På denne måde kan der opbygges en rapport med flere undersøgelser, end dem som projektdeltagerne selv er i stand til at foretage. Relevant information og teori kan ligeledes opnås gennem denne metode, ved at analysere videnskabelige dokumenter eller bøger for at udlede forståelse og viden fra dokumenterne. 
	
\chapter{Problemanalyse}
Daniels del:
Små virksomheder vælger i dag ikke at udvide pga. manglende overskuelighed ift. administration. Flere små virksomheder frygter at miste overblikket over deres administration, hvis de udvider deres virksomhed og ansætter flere. De vælger altså i stedet at beholde deres størrelse, som en virksomhed på én til fire personer [2]
I 2012 blev der indført en ny lov, som skræmte små virksomheder yderligere fra at udvide. Denne lov ville give virksomheder, som ikke havde administrationen i orden, bøder. Denne lov rammer især de små virksomheder [1].
Et eksempel på en type virksomhed, som ikke udvider, er håndværksvirksomheder. Tre ud af fire af disse virksomheder har kun én til fire ansatte. Dette skyldes som nævnt før, at de frygter administrationen af flere ansatte end dem de allerede har. De fleste af disse virksomheder er ofte ejet af en dygtig håndværker, hvor administration ikke er en del af personens kompetencer [2].

[1] Nye administrative bøder på vej til virksomhederne
\url {http://www.fsr.dk/Nyheder%20og%20presse/Pressemeddelelser/Pressemeddelelser%202012/Nye-administrative-boeder}

[2] Små firmaer orker ikke at voske
\url {http://www.licitationen.dk/smartphone/artikel/VisArtikel.aspx?SiteID=LI&Lopenr=105190076}

Den del Daniel står for:
I den seneste tid er det blevet vigtigere for virksomheder at benytte IT til administration. Derfor er der også flere virksomheder, som benytter sig af forskellige IT services. Uanset sektor og branche sker der en stigning i brug af IT, se figur XXX.

Billede fra [1, s. 20]
På figur XXX vises det, at over halvdelen af virksomhederne, bruger cloud computing til ting som filhåndtering og database brug. Yderligere vises det at over en tredjedel benytter det til infrastruktur, kundedata, økonomi med mere. 

Billede fra [1, s. 21]

Ud fra disse statistikker er det tydeligt at, der sker en stigning  i virksomheders brug af IT. SKAL VIDERESKRIVES

Kilde 1 : \url{http://www.dst.dk/pukora/epub/upload/19252/itvirk.pdf}

\newpage

Arbejdsmiljø overordnet

Et arbejdsmiljø er et samspil af de sammenhænge, påvirkninger og forhold som mennesker arbejder under. Et arbejdsmiljø er også den tekniske og sociale udvikling af arbejdspladsen, hvilke kan være med til at styrke den enkeltes sikkerhed samt styrke menneskets fysiske og psykiske sundhed.

Et arbejdsmiljø kan have mange forskellige påvirkninger på arbejderen. Eksempler på disse kan være: 
Psykisk påvirkning:

Fysisk påvirkning:
\begin{itemize}

\item Slid: Et arbejde kan være fysiskt hårdt på mange måder. Man har f.eks. fået tildelt en ubehagelig stol som på længere sigt kan resultere i rygsmerter. Dårlige muse anordninger til computerbrug kan også resultere i skader i håndled. 
\item Støj: En larmende arbejdsplads kan også have stor indvirkning på arbejderens hørelse, i form af fysiske høreskader, men er også en faktor til psykisk stress.
\item Klima (Temperatur, frisk luft): Aspekter som for høj eller for lav temperatur(højere end 25 grader, lavere end 20 grader), og et indelukket eller forurenet klima kan have fysisk påvirkning på medarbejderen i form af f.eks. tørre øjne, træthed og hovedpine. 
\end{itemize} 
Social påvirkning:

\url{http://www.arbejdsmiljoviden.dk/Viden-om-arbejdsmiljoe/Hvad-er-arbejdsmiljoe}
\url{http://www.arbejdsmiljoviden.dk/Viden-om-arbejdsmiljoe/Fysisk-arbejdsmiljoe}
\url{http://www.arbejdsmiljoviden.dk/Viden-om-arbejdsmiljoe/Stoej}
\url{http://www.arbejdsmiljoviden.dk/Viden-om-arbejdsmiljoe/Indeklima}

\newpage

For at forebygge konsekvenserne af et dårligt arbejdsmiljø, er en struktureret eller fleksibel vagtplan en mulighed.
En fleksibel vagtplan, hvor medarbejderen er med til at vælge sine vagter, vil udover at skabe en større medarbejdertilfredshed også ansvarliggøre medarbejderen, sikre et bedre arbejdsmiljø, ved at skabe større fleksibilitet mellem arbejde og privatliv og forbedre ressourceudnyttelsen og kvaliteten af arbejdet. Det øgede ansvar giver også medarbejderen en forståelse for arbejdsplanlægningen og dermed et større engagement i at få arbejdsplanen til at gå op.
Samtidig vil vagtplanlæggeren få mere tid til andet arbejde som f.eks. kvalitetssikring.
Undersøgelser viser også, at gode og fleksible arbejdstider er fastholdelsespotentiale for medarbejderne, hvilket vil sige, at kontinuiteten opretholdes.
\\
\url{http://www.regioner.dk/~/media/Publikationer/Regionerne%20som%20arbejdsgivere/Arbejdsmilj%C3%B8%20sygefrav%C3%A6r%20og%20trivsel/Rapport%20Arbejdstid%20medbestemmelse%20arbejdsmiljo%20pdf.ashx}
\\\\
Ved at medarbejderne er med i vagtplanlægningen, vil der også være mindre behov for afløsere, da de selv bestemmer deres vagter i et vist omfang.
\\
\url{http://www.arbejdsmiljoviden.dk/Aktuelt/Se-hvad-andre-goer/Ny-fleksibel-arbejdstid-gav-mere-arbejdsglaede---Kirke-Stillinge-Plejecenter}

\newpage

Tamigo
Tamigo er et online vagtplanlægningsværktøj som har flere forskellige funktioner. Disse funktioner tillader bl.a. administratoren at planlægge effektivt efter de gældende timeregler. Deltidsarbejderne har desuden mulighed for at bytte deres vagter og anmode om fridage online sådan at arbejdsgiveren kan bevarer overblik.

Pros
\begin{itemize}
\item Tæller medarbejdernes timer og advarer arbejdsgiveren hvis antallet overskrides eller undermineres.
\item Vagtbytnings funktion som har indbygget dokumentation af ændringerne.
\item Apps til smartphones som tillader medarbejderne at se planen online.
\item Programmet og appen indeholder en telefonliste over medarbejderne som gør kontakt mellem arbejder og arbejdsgiver nemmere.
\end{itemize}
Cons
\begin{itemize}
\item Kan ikke ses uden internetforbindelse.
\item Kan virke uoverskuelig hvis ikke man er blevet sat ind i systemet.
\end{itemize}
Kilder:
\\
\url{http://www.tamigo.dk/}
\\
\url{https://www.trustpilot.dk/review/www.tamigo.dk}
\\
\url{https://play.google.com/store/apps/details?id=tamigo.android.activities}
\\
Planday

Planday har både positive og negative aspekter. Anmeldelserne der er blevet lavet omkring Planday er både positive og negative. Planday har valgt at bruge disse anmeldelser til at forbedre deres produkt. Hver gang de får en god anmeldelse takker de for den gode anmeldelse. Men når de får en dårlig anmeldelse, svarer de tilbage de tilbage med en hjælpende kommentar, der kan være med til at hjælpe brugeren.

Pros:
\begin{itemize}
\item Kan bruges af alle onlineapparater som har adgang til internettet. 
\item Planday vagtplanlægning funktioner dækker over aspekter som:
\item Tilgængelighed: Alle i virksomheden har adgang til vagtplanlægningen.
\item Vagtbytning: Man kan f.eks. melde sig syg. Når man har meldt sig syg så kan vagtplanlæggeren se at man er syg, og kan med et enkelt klik se ledige medarbejdere med samme kvalifikationer som den syge, der kan tage den syges vagt. Derved bliver der enten sendt en mail eller en besked til den medarbejder der skal acceptere vagten. 
\item Kommunikation: Medarbejderne kan kommunikere internt gennem programmet.
\item Kontrol af timer og omkostninger (Beregner hver måned hvor meget den ansatte skal have i løn).
\end{itemize}

Cons:
\begin{itemize}
\item Browserversionen er god, men appen har fået negativ feedback.
\item Den crasher ved opstart.
\item Opdateringerne virker ikke til at hjælpe på problemet.
\item Utilregnelig: Lukker ofte sig selv, eller fryser fast ved opstart.
\item Kan ikke vise de beskeder man har modtaget.
\item Man kan ikke se om ens ferie er godkendt.
\item Kan ikke tilgås uden et online apparat
\end{itemize}


\url{http://www.fdih.dk/medlem-netvaerk/medlemsoversigt/p-s/planday-as/}
\url{http://finans.dk/live/erhverv/ECE7266963/Dansk-app-succes-har-sikret-sig-%C2%BBet-scoop%C2%AB/}
\url{http://planday.dk/}
\url{https://play.google.com/store/apps/details?id=com.planday.ninetofiveapp&hl=da}


lectio.
lectio er et planlægningsværktøj som primært bliver brugt på gymnatier mellem lærere og elever. Servicen gør det muligt for lære at lave et tilrettelagt skema som eleverne kan tjekke online. udover det kan lærerne også dynamisk aflyse og flytte lektioner. Lectios anden primære funktion gør det muligt for den enkelte elev at uploade opgaver til Lectios servere hvor der bliver holdt styr på elevens totale afleveringsprocent samt fysisk fravær.

\url{http://www.macom.dk/publish/da/lectio.htm}

	\section{Interessentanelyse}
		\subsection{Interessenterne}
	\section{Interview}
		\subsection{Interviewspørgsmål}
		\subsection{Interview med arbejdsgivere}
		\subsection{Interview med ansatte}
	\section{Nuværende løsninger}
	\section{Helhedsvurdering}
	\section{Problemafgrænsning}
	\section{Problemformulering}

\chapter{Problemløsning}
	\section{Kravspecifikation}
	\section{Teori}
	\section{Design}
	\section{Test}
	\section{Implementation}

\chapter{Konklusion}

\chapter{Diskussion}

\chapter{Refleksion}
		
\chapter{Litteraturliste}
\url{http://library.au.dk/guides/opgaveskrivning/kildekritik/}

Power i projekter og portefølje, Mette Lindegaard Attrup og John Ryding Olsson, 2008, Jurist- og økonomforbundets forlag.

\url{http://gymportalen.dk/politiksdan/15203}

Den Store Danske, Interview, Bjarne Hjorth Andersen og Erik Lund, 12-02-2015
\url{http://www.denstoredanske.dk/Samfund,_jura_og_politik/Massemedier Journalistiske_genrer_og_stofomr%C3%A5der/interview}

\url{http://teknologi.systime.dk/index.php?id=568}

\url{https://www.moodle.aau.dk/pluginfile.php/394871/mod_resource/content/1/Teknologivurdering%20SW%20DAT%202014.pdf}

\url{http://www.aau.dk/om-aau/aalborg-modellen-problembaseret-laering}

\url{http://www.aau.dk/digitalAssets/62/62748_17212_dk_pbl_aalborg_modellen.pdf}

\chapter{Bilag}

\end{document}