
%\section{Hvorfor mindre virksomheder ikke udvider}
%Der eksisterer en del mindre virksomheder indenfor byggebranchen, mange af dem vælger ikke at blive større, når håndværkermesteren indtjener det han gerne vil. Udover det virker administrationen for mange af de små virksomheder som uoverskuelig. De vælger ikke at udvide pga. manglende overskuelighed med hensyn til administration. Flere af dem frygter at miste overblikket over deres administration, hvis de udvider deres virksomhed og ansætter flere. I stedet vælger de at beholde deres størrelse, som en virksomhed på én til fire personer %\citep{SmaaFirmaerOrker}.
%I 2012 blev der indført en ny lov, som skræmte små virksomheder yderligere fra at udvide. Denne lov ville give virksomheder, som ikke havde administrationen i orden, bøder.\todo{Skal der bare stå bøder eller skal det omformuleres?} Denne lov rammer især de små virksomheder %\citep{Nyeadmboder}.
%Et eksempel på en type virksomhed, som ikke udvider, er håndværkervirksomheder. Tre ud af fire af disse virksomheder har kun én til fire ansatte. Dette skyldes som nævnt før, at de frygter administrationen af flere ansatte end dem de allerede har. De fleste af disse virksomheder er ofte ejet af en dygtig håndværker, hvor administration ikke er en del af personens %kompetencer \citep{SmaaFirmaerOrker}. \todo{Afsnittet skal generelt kigges på}


Fungerer sgu ikke helt til interessent analysen

Lange køer kan f.eks. have stor indflydelse på kunden, idet at kunden skal bruge længere tid i f.eks. et supermarked, end kunden skulle gøre hvis der ikke var nogen kø \citep{Brix2012}. Sådan et problem ville kunne løses med ordentlig arbejdsplanlægning, der satte nok medarbejdere de rigtige steder i supermarkedet, for at skabe en hurtig kundeekspedition. 