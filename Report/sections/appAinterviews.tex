\chapter{Interviews}\label{interviews}
\section{Interview med Benjamin fra Subway}\label{app:subway}
Her interviewer Jesper og Jens, Benjamin fra Subway.\\
Jesper: ”Jammen vi sidder her med Benjamin fra subway og du er butikschef her?”\\
Benjamin: ”Jeg er resturantchef her”\\
Jesper: ”Resutantchef ja okey”\\
Jesper: ”Og vi vil gerne høre, har i et elektronisk eller vagtplan-sss-sss når i laver vagtplaner og sådan noget”\\
Benjamin: ”Altså det vi gør,vi har jo ikke så mange medarbejdere så vi laver egentlig bare selv, vi har, jeg har selv lavet et excel ark”\\
Jesper: ”Okey”\\
Benjamin: ”Hvor vi bare smidder det ind”\\
Jesper: ”Ja okey”\\
Benjamin: ”Så kører vi lidt nogle faste vagter der altid er”\\
Jesper: ”Ja oh”\\
Jens: ”Ja okey”\\
Benjamin: ”og så ændre det sig bare lidt fra uge til uge”\\
Jesper: "Ja"\\
Jens: "okey emm"\\
Jens: ”Så vi har ikke noget ligesom som f.eks McDonalds har noget sådan hvor de kan plotte ind, alle de her, hvornår hvem skal have fri hvornår”\\
Jesper: "Ah"\\
Benjamin: "Så når de har en vagt så kan de gå ind og så vælge, så kan de se hvad for nogle medarbejdere der er grønne f.eks. (?)"\\
Jens: "Mhh"\\
Jesper: "Okey"\\
Jens: "Mhh"\\
Benjamin: "Så der er egentlig lidt mere arbejde i det fordi man skal sidde og holde styr hvem skal holde fri de dage her og sidde og skrive det ind"\\
Jesper: "Ja okey"\\
Benjamin: "Men lige nu her når vi ikke har... hvad har vi en femten medarbejdere? det er lige før det er lige på grænsen til at det kan betale sig at man rent\\ faktisk køber et eller andet program" \\
Jesper: "Ja"\\
Benjamin: "Der koster helt vildt meget"\\
Jesper: "okey"\\
Jens: "Ja ja okey"\\
Jens: "Fordi altså når man tænker når i nu skal have kontakt til din medarbejder, du skal fortælle dem når der er en ændring og sådan noget der så skal du til at have fat i dem over telefon?"\\
Benjamin: "Der har, der har vi godt nok facebook gruppen selvfølgelig"\\
Jens: "Så har i en facebook gruppe hvor i skriver det derind?"\\
Benjamin: "Hvor vi skriver hvis der er noget"\\
Jesper: "Ja okey"\\
Jens: "Ja okey så i har slet ikke nogen, nogen vagtplanløsning det i ligesom har det er ligesom det, det er selvlavet excel"\\
Benjamin: "Ja det er lidt hjemmelavet og syet sådan lige ehhh..."\\
Jens: "Ja præcis"\\
Benjamin: "Os så fordi så er det let lige at lave om hvis man lige vil lave det lidt anderledes og sådan noget"\\
Jesper: "Ja"\\
Benjamin: "Selvfølgelig gør det måske lidt lette at have et eller andet program ehh.. men"\\
Jens: "Ja"\\
Benjamin: "I ved hvor meget det koster det er jeg sikker på"\\
Jesper: "Ja selvfølgelig"\\
Benjamin: "Det er helt vildt dyrt"\\
Jesper: "Ssssh"\\
Benjamin: "Så der vil jeg sige at man helst skal have de der halvtredt hunderede medarbejdere før, sådan noget det kan betale sig"\\
Jesper: "Ja"\\
Jens: "Ja, jae eller så skal man være en del af en større kæde ikke?"\\
Benjamin: "Ja"\\
Jens: "Man kunne sige..."\\
Benjamin: "Man kunne sige hvis vi var lidt større i Danmark så kunne det også være at der var et eller andet"\\
Jesper: "Ja"\\
Benjamin: "Men nu er der kun tre subways i Danmark"\\
Jesper: "Okey"\\
Benjamin: "Ja så vidt jeg ved så tror jeg ikke subway har noget som, på verdensplan, program"\\
Jesper: "Nej nej det ved vi egentlig heller ikke"\\
Jesper: "Så hvor mange timer ligger du i at lave en vagtplan?"\\
Benjamin: *pust ud\\
Benjamin: "Det kommer lidt an på hvor meget fri-ønske der er sådan selvfølgelig som sommerferie"\\
Jesper: "Ja"\\
Benjamin: "Der tager det meget lang tid"\\
Jesper: "Det er klart"\\
Benjamin: "Fordi der er mange der skal have fri i forhold til en almindelig november hvor der ikke er nogen der skal noget som helst"\\
Jesper: "Ja"\\
Jens: "Mhh"\\
Benjamin: "Men jeg ville, jeg vil nok skrive, jeg bruger nok en dag på det"\\
Jesper: "Okey"\\
Benjamin: "Det kan godt være"\\
Jesper: "Ja okey"\\
Benjamin: "Så det tager alligevel noget tid når man ikke har sådan et der er færdigt i forvejen"\\
Jesper: "Jaer jaer"\\
Jens: "Mhh"\\
Benjamin: "Det er selvfølgelig det, det eneste der er trælst ved det det tager idt tid men der er ikke sådan..."\\
Jesper: "Nej og så medarbejdernem de får bare et papir eller sådan med deres vagtplan?"\\
Benjamin: "Vi har den faktisk kun i excel på facebook"\\
Jesper: "Okey så ikke, sådan man kan se den derinde"\\
Benjamin: "Så lægger jeg den op der og så kan de se den der"\\
Benjamin: "Vi har også haft den til at være printet ud fordi det var rart nok at man kunne se det men nu er det kommet dertil hvor alle har facebook"\\
Jesper: "Ja ja lige præcis"\\
Benjamin: "så ja, så er det bare ligemeget"\\
Jens: "Ja"\\
Jens: "Ja man ser noget på Facebook så ved man også at alle får det"\\
Jesper: "Ja"\\
Jens: "Så er du ikke i tvivl om at..."\\
Benjamin: "Man kan se at de allesammen har set det"\\
Jesper: "Ja"\\
Benjamin: "Man kan ikke komme og sige at man ikke har set det og så står der at han har set det"\\
Jens: "Mhh Mhh"\\
Jesper: "Ja hehe okey"\\
Jesper: "Godt nok"\\
Jesper: "Også når de skal bytte vagter det foregår også bare over facebook?"\\
Benjamin: "Ehh der har vi sådan en mail hvor vi sådan skemamail hvor de kan sende deres friønsker og hvis de går til noget eller hvis de bytter"\\
Jesper: "Ja"\\
Benjamin: "Så sender de den bare til den altså"\\
Jens: "Og det samme det gælder også med sygemeldninger?"\\
Benjamin: "Ehh"\\
Jens: "Eller hvad?"\\
Benjamin: "Der ringer man så ind"\\
Jesper: "Ja"\\
Jens: "Okey der ringer du ind og når du så skal have en til at tage vagten er det så på Facebook eller er det på mail igen?"\\
Benjamin: "Ehhh hvis man er syg så skal man faktisk... så har vi i hvertfal i weekenderne en der har ansvaret for at få dækket vagten ind"\\
Jesper: "Okey"\\
Benjamin: "Så hvis man er syg så skal man egentlig ikke gøre noget man skal egentlig bare melde at man er syg"\\
Jesper: "Ja"\\
Benjamin: "Ehh men ellers hvis de vil bytte en vagt så gør de det bare inde på facebook senere om der er nogen der vil tage hans vagt"\\
Jesper: "Okey"\\
Benjamin: "Jeg plejer så at sige at det er en dårlig ide fordi folk er dårlige til at svare"\\
Benjamin: "Når man bliver ringet op og siger kan du arbejde i morgen?"\\
Jens: "Ahh så kan man godt lige"\\
Benjamin: "Ja okey"\\
Jens: "Det er sværre at sige nej til..."\\
Benjamin: "Men står det inde på facebook kan du ikke lige arbejde? Ejj det gider jeg egentlig ikke, jeg svarer ikke på den"\\
Jesper: "Nej nej lige præcis"\\
Jens: "Det er lidt sværre lige at svare på en der ringer til en"\\
Jesper: "Ja okey"\\
Benjamin: "Jammen jeg skal virkelig noget jeg kan bare ikke, okey"\\
Jesper: "Var der mere vi skulle have Jens?"\\
Jens: "Nej det er fair nok"\\
Jesper: "Vi skal bare lige være sikker, er det oket at vi bruger det her i vores rapport ikke?"\\
Benjamin: "Jo jo"\\
Jens: "Det er ikke noget kommercielt brug overhovedet det er bare til en rapport"\\
Jesper: "Universitetet"\\
Benjamin: "Det er i velkommen til"\\
Jesper: "Det var det"\\
Benjamin: "Super"\\
Jens: "Ja tak for det"\\

\section{Interview med Henrik Kluhjef fra Netto}\label{app:netto}
Her interviewer Andreas og Ronni, Henrik fra Netto.\\
Andreas: “Jeg hedder Andreas Lauritsen og du hedder?”\\
Henrik: “Henrik Kluhjef”\\
Andreas: “Yes, og det er i orden at vi bruger den her til vores studies rapport?”\\
Henrik: “Det er helt iorden”\\
Andreas “Yes, vi vil gerne høre noget omkring jeres vagtplan system øhm hvordan fungere jeres nuværende vagtplan system?”
Henrik: “Jamen vores vagtplan system fungere på en måde at jeg laver en fast 4 ugers arbejdsplan”\\
Andreas: “Ja”\\
Henrik: “Som så køre igen og igen øøøh i en 4 ugers periode så man hele tiden har de samme”\\
Andreas: “Så man har de samme vagter i ugen eller hvad?”\\
Henrik: “Ja øhm så det er de samme der har mandag etc”\\
Andreas: “Og det system fungerer det elektroniskt eller skal du selv sætte det ind eller skrive det ned”\\
Henrik: “Jeg laver sådan set bare 4ugersplanen på computeren og så kopiere jeg bare ugerne ind i øhh i- i kalender året og så køre det bare igen og igen og igen”\\
Andreas: “Øhh hedder det er det et specifikt program eller det det”\\
Henrik: “Det hedder teameplan ja”\\
Andreas: “Ok. og øhh det er noget i bruger i hele netto eller er det bare i denne afdeling?”\\
Henrik: “Det er hele netto”\\
Andreas: “Ok øøøhm fx øh øh er systemet fleksibelt fx hvis der er nogen der bliver syge eller der er forhindrede i at møde op?”\\
Henrik: “Det er det hele der kan man ligge sygdom eller ferie, det hele. kurser og barsel man kan ligge alt ind.”\\
Andreas: “Ok”\\
Henrik: “De regner lønnen ud også”\\
Henrik: “Hvor meget man ligger til at bruge om måneden på året og på ugen og på dagen, alt. og selv for at man får heligdags tillæg øøh aften tillæg jeg skal ikke ligge noget som helt ind”\\
Henrik: “Det er systemet selv sat op til jeg kan ikke gøre noget forkert så længe medarbejderen scanner når det kommer og går”\\
Andreas: “Ok og det skal de selv stå for?”\\
Henrik: “Det skal de selv stå for og det er deres eget ansvar og så kan de også selv stå for at deres løn passer hvis de scanner på deres kort hver gang de kommer og hver gang de går jamen så er der aldrig noget problem”\\
Andreas: “Ok”\\
Andreas: “Øhmm har du lagt mærke til nogen problemer med programmet er der noget du synes er galt med det eller noget der måske mangler?”\\
Henrik: “Nej der er der ikke. nu har jeg kun arbejdet i netto i nogle år”\\
Andreas: “Ok”\\
Henrik: “Men jeg syntes det fungere meget godt det system vi har, det har ligesom de funktioner som det skal have og man kan gå ind i de funktioner og se den økonomi del af det som man især har brug for i min stilling”\\
Ronni: “Er der der nogen der har haft indvendinger til den her vagtplan”\\
Henrik: “Jamen det afhænger ligesom af butiks chefen det det handler ikke så meget om butikschefen, det handler mere om de aftaler man har med butikschefer og medarbejdere imellem om man er tilfreds eller utilfreds. system mæssigt der ved de os godt selv om de kommer og når de går så ved de også godt at så passer deres løn”\\
Andreas: “Ok øhmm når den her plan skal sættes op tager det så lang tid eller kan der gøres meget hurtigt”\\
Henrik: “Jamen det kommer an på hvis der er mange medarbejdere der har sagt op så jeg har ansat nogle nye så skal jeg jo ændre arbejds tiderne imellem de forskellige linier det kan godt tage et par timer men lige så snart det er gjordt altså selve passis planen er på plads så kopiere jeg bare ind og det tager kun er par minutter bum! så er det klart”\\
Andreas: “Ok”\\
Andreas: “Øøøhm når så folk vil have byttet vagter hvordan fungerer det så?”\\
Henrik: “Jamen jeg giver så lang snor herinde til mine medarbejdere at de kan selv bytte så meget de vil øøh så længe der kommer en og så er der igen jamen så kan de godt være at de har byttet sig til en vagt og det ikke står på vagt planen med så når de scanner for alligevel deres løn”\\
Ronni: “Hvordan kan de se at de har en vagt?”\\
Henrik: “Det kan de sagtens i snit så laver jeg en 4-4 måneders vagtplan som jeg skriver ud til dem og et par uger inden de udløber så laver jeg nogle nye og høre om der er nogen der har nogle ønsker og om de gerne vil ændre nogle ting det skal selvfølgelig passe til mig men det skal også passe til dem så de ikke føler sig utilfredse. og så får de en vagtplan i hånden og så kan de jo selv se hvornår de skal arbejde øhmm så det er de aldrig i tvivl om”\\
Andreas: “Ok jeg har lige et spørgsmål mere når de så bytter vagt har du så nogen måde at se på hvem der dukker op?”
Henrik: “Det har jeg ikke men jeg har stor tillid til mine medarbejdere, hvis de træder ved siden af”\\
Andreas: “Ok”\\
Henrik: “Så skal jeg nok fortælle dem det, men jeg vil gerne selv fortælle min chef, så jeg har det sådan at så vil jeg også gerne selv fortælle dem om de fortjener min tillid”\\
Andreas: “Ok”\\
Andreas: “Men jeg tror det var det hele, mange tak”\\


\section{Interview med Kasper Kanstrup fra KIWI}\label{app:kiwi}
Her interviewer Andreas og Ronni, Kasper fra KIWI.\\
Andreas: Jeg hedder Andreas Smed Lauritsen og du hedder?”\\
Kasper: "Øh, Kasper Kanstrup.”\\
Andreas: "Yes, og vi må gerne bruge det her interview i sammenhæng med vores rapport?”\\
Kasper: "Ja det er i velkommen til.”\\
Andreas: "Yes, øhm, hvis vi nu bare spørger omkring jeres vagtsystem, har i et vagtsystem, i for, er det elektronisk eller er det på papir?”\\
Kasper: "Ja vi har et elektronisk program der hedder Staffweb.”\\
Andreas: "Staffweb?”\\
Kasper: "Ja.”\\
Andreas: "Okay.”\\
Kasper: "Hvor man kan se vagterne både på via internettet, men også via et program som ligger på computeren.”\\
Andreas: "Yes, øhm, er det et fleksibelt system i forhold hvis der nogle der bliver syge eller der er nogen der ikke, der er forhindret i at møde op, hvad gør i så?”\\
Kasper: "Ja så kan man bare gå ind og rette i det hvis man har programmet, som vores butikschef har også har vi det på computeren for arbejdet.”\\
Andreas: "Yes.”\\
Ronni: "Øhm, har i oplevet nogen positive eller negative ting ved at bruge det her?”\\
Kasper: "Øhm, altså i forhold til før hvor vi havde det på papiret, der er det nemmere, fordi enhver har ikke undskyldningen med, min vagtplan er blevet væk, fordi man kan altid bare gå ind på nettet, også trykke, og for det er en simpel kode og et simpelt brugernavn som er bestemt af firmaet, så man kan altid finde det, så det er en klar fordel i forhold papir vagtplaner, bare det at det ikke kan blive væk, man har aldrig undskyldning, det vidste jeg ikke.”\\
Ronni: "Hvordan gør man f.eks. hvis man gerne vil bytte en vagt f.eks. med en anden en?”\\
Kasper: "Så skriver man til den ansvarlige øh, og skriver jeg byttet vagt med den person den dag, fra kl. til det til det, og det skal så foregå inden kl. 17, så man skal manuelt ind og rette det øh, hvis man har, man kan kun gå ind og rette hvis man ans, hvis man har en kode og dermed ansvaret for vagter.”\\
Ronni: "Gør man det så først internt mellem de andre eller fraskriver man sig bare vagten?”\\
Kasper: "Altså hvis det er, hvis det er for mere end 4 uger frem, så har man ret til bare at sige, jeg kan ikke arbejde den dag, hvis det er med mindre end 4 ugers varsel, så skal man nå at bytte vagten.”\\
Andreas: "Okay, så har i, når man så har byttet vagt, så har i en dokumentation for hvem det er der møder op?”
Kasper: "Ja vi skal have, vi skal have det på, vi har sagt, vi vil ikke have det mundtligt, det er chefen og mig der står for det”\\
Andreas: "Ja.”\\
Kasper: "Vi vil have det på sms, sådan så at vi kan sige, du skrev til mig og det er sådan det er, så kan vi altid have det.”\\
Ronni: "Hvad gør man så i tilfælde af at der ikke møder op?”\\
Kasper: "Øhm, ringer til dem.”\\
Ronni: "Ja.”\\
Kasper: "Også høre hvorfor de ikke er på arbejde.”\\
Ronni: "Okay.”\\
Andreas: "Okay, er der nogle problemer i har haft med systemet?”\\
Kasper: "Ikke umiddelbart nej.”\\
Andreas: "Der er ikke noget i umiddelbart mangler ved systemet, som kunne være?”\\
Kasper: "Der mangler, systemet mangler den ting, at når en medarbejder stopper, at man så kan fortsætte vagtplanen i en anden persons navn, du kan ikke bare rykke, sige alle den persons vagter, skal flyttes over til den person, det synes jeg at systemet mangler, fordi det ville være mange ting nemmere.”\\
Ronni: "Ja.”\\
Andreas: "Okay.”\\
Ronni: "Umiddelbart så har vi ikke mere."\\
Andreas: "Nej det er relativ kort."\\


\section{Interview med Kasper fra Home}\label{app:home}
Her interviewer Luca, Kasper fra Home.\\
Luca: “Og mit navn er Luca og jeg sidder her med Kasper, som du er chef for Home, ikke?”\\
Kasper: ”(??? For butikken her ??”?)\\
Luca: “For butikken her?”\\
Kasper: ”Ja”\\
Luca: ”Og i benytter ikke noget IT-system til at styrer vagtplanlægning?”\\
Kasper: ”Nej, vi har vores almindeligt Outlook, der kører og så har vi, vi kører sammen med Hasseris afdelingen derude”\\
Luca:”Ja”\\
Kasper:”og så kører vi på og så har vi hver tredje weekend, øhh, og det putter vi bare ind i kalenderen.”\\
Luca: “Okay”\\
Kasper: “Og det bliver lavet i en manuel kalender først, så bliver det bare proppet ind i kalenderen deroppe”\\
Luca: “Så der er bare en Outlook-kalender der bare synkroniserer med alle sådan set”\\
Kasper: “Lige præcis”\\
Luca: “Okay”\\
Kasper: “Så det er faktisk så simpelt som det overhovedet kan være”\\
Luca: “Ja”\\
Kasper: “Ehmm”\\
Luca: “Men så i trods alt en IT-løsning”\\
Kasper: “Ja, det gør vi, vi bruger Outlook, (????) vi skriver det selv ind, det er ikke noget hvor der er at der en eller anden fin vagtplan med en eller anden fin…”\\
Luca: “Nej nej”\\
Kasper: “løsnings modul eller andet i”\\
Luca: “Nej nej, okay. Hvad så når i bytter vagter og sådan noget, så skriver i bare det op, at i skal..?”\\
Kasper: “Så ændre vi det bare”\\
Luca:”Så ændre..”\\
Kasper: “ (???) inde i Outlook, så går vi manuelt ind og ændre det.”\\
Luca: “Okay”\\
Kasper: “Og skriver at eh, nu har Helene vagten i stedet for Kasper”\\
Luca: “Og det virker.. fungere fint. Der er ikke noget der? Det virker fint for jer?”\\
Kasper: “Ja”\\
Luca: “Og hvor mang-”\\
Kasper: “Der er ikke sket noget endnu, i hvertfald”\\
Luca: “Nej, okay, Det er jo helt fint. Hvor mange ansatte er det så i er? Sådan, eller hvor mange er det der benytter det her system.”\\
Kasper: “Det er jo alle, alle Mortens butikker, altså han har fire forretninger her i Aalborg, og så har han ehh, (???? (en Skive?)) i Thisted og så har han noget (???), nej i Blokhus oh ehh.”\\
Luca: “Okay”\\
Kasper: “Så der er ehh, men vi arbejder så sammen med, vores vagt-( uge??) kører på skift (???).”\\
Luca: “Ja”\\
Kasper: “Tænker jeg, og (???) skiftes internt”\\
Luca: “Okay”\\
Kasper: “Så vi har ikke noget med de andres vagter at gøre”\\
Luca: “Nej”\\
Kasper: “Som sådan, men alle sammen vi kører bare Outlook”\\
Luca: “I kører bare Outlook?”\\
Kasper: “Ja”\\
Luca: “Det er også helt fint jo, jamen, så er der faktisk ikke så, der er faktisk ikke så meget mere vi kan spørge om. (?????).. spørge ind til systemet, for vi ved jo hvad Outlook det er og vi kender jo til det. Så der er jo ikke så meget der”\\
Kasper: “Så der er jo desværre ikke så meget kød på herinde fra”\\
Luca: “Nej det er helt i orden, men det er også derfor vi er ude at undersøge, så det er helt fint”\\

\section{Interview med Dennis fra Rema 1000} \label{app:rema}
Her interviewer Luca og Ronni, Dennis fra Rema 1000.\\
Luca: Yes det, vi sidder Luca og Ronni fra Aalborg Universitet og vi sidder her med?\\
Dennis: Dennis fra Rema 1000 i Nørresundby.\\
Luca: Yes og vil gerne hvad for et system i bruger til vagtplanlægning.\\
Dennis: Vi bruger Tamigo.\\
Luca: Okay har i nogle problemer med det vagtplanlægningssystem eller fungerer det bare helt fint?\\
Dennis: Det fungerer rigtig fint.\\
Luca: Der, der er ikke nogen ting i oplever som kan være træls ved programmet?\\
Dennis: Nej, ikke hvad jeg kender af.\\
Luca: Okay, hvad er det der virker rigtig godt ved programmet?\\
Dennis: Det er overskueligt, det er funktionelt, vedrørende både med telefonnumre, oplysninger på kollegaer og medarbejdere, det er nemt at se sine vagter, det er nemt at bruge, det er hurtigt og enkelt, det fungerer både fra computer og fra mobiltelefon, så ja.\\
Luca: Hvordan fungerer det så når folk skal bytte vagter, kontakter de stadigvæk dig, eller kan de lægge noget ind i det der vagtplanlægningssystem eller Tamigo som det så hedder?\\
Dennis: Det fungerer at man har sine vagter, og hvis man vil af med en vagt så udbyder man den, også kan kollegerne så byde på den og det skal jeg så godkende når der kommer et vagtbytte kan man sige, og det foregår også bare på telefonen og det tager lige nøjagtig tre sekunder, der vurderer jeg så om, jeg har nogle over 18 og nogle under 18 og det sker jo engang imellem at der er en under 18 der byder på en over 18 vagt og det duer ikke, så bliver den bare afvist og så er det det, ellers så kører det bare igennem.\\
Luca: Okay.\\
Ronni: Er der så nogle fordele med oversigt over ting, f.eks. som løn?\\
Dennis: Ja, det bruger jeg også, jeg laver min derinde, og der er masser af funktioner på det her Tamigo som vi slet ikke bruger, men jeg registrerer timerne og eksporterer dem ned til min lønafdeling som så står for at få udbetalt det rigtige i løn.\\
Ronni: Men er der stadig noget arbejde for dig eller klarer Tamigo det meste for dig.\\
Dennis: Det klarer det meste af det, altså hvis jeg opretter medarbejderne rigtigt med rigtige lønkoder og de rigtige timer så kører det bare.\\
Ronni: Skal det opdateres eller kører det bare?\\
Dennis: Det foregår bare med opdateringer om natten hvis det er sådan, og når der skal laves løn engang om måneden så henter lønafdelingen det, det er ikke noget jeg mærker til, jeg skal ind og lukke dagene til et givet tidspunkt, og det er egentlig bare det, meget enkelt for mig.\\
Ronni: Hvis en medarbejder f.eks., jeg går ud fra at det er elektronisk?\\
Dennis: Ja det er det.\\
Ronni: Hvis en medarbejder ikke har mulighed for at komme på internettet, hvordan sikrer i så at vedkommende kan se vagtplanen?\\
Dennis: Jeg kan sende, hvis i et sådant tilfælde, langt ud, der kan jeg sende vagtplanen per sms og på mail til personen, jeg har faktisk haft som ikke havde en smartphone, så kan man så ikke bruge den funktion, så kan man bruge internetdelen og den kunne man evt. bruge når man mødte ind på vagt, så kunne man bruge min butiks PC til at printe dem ud.\\
Luca: Okay, hvordan fungerer det sådan med hensyn til når der kommer nye medarbejdere hvis der er nogen der siger op eller der er folk der ikke skal arbejde her mere?\\
Dennis: Jamen de bliver bare afmeldt kan man sige, og jeg plejer at bruge, der er en informationsside når man logger ind i det her, hvor man kan skrive hvis der er nogle personalearrangementer eller nogen der stopper eller nogen der begynder så informerer jeg om at der er en ny medarbejder der starter der og har måske erfaring fra en Rema 1000 jamen så er det en der går direkte eller en der skal have noget oplæring.\\
Ronni: Hvor lang tid tager det at oprette denne person så, inde i systemet?\\
Dennis: Det tager et minut.\\
Luca: Okay.\\
Dennis: Ja, det er skåren helt, altså det er lavet så simpelt, så jeg bruger ikke meget tid på, der er de informationer der skal være, og der ligger rigtig mange funktioner bagved stadigvæk som vi ikke bruger, men der er muligheder for hvis man bruge det endnu mere grundigt at man, der er en masse funktioner som jeg ikke bruger, fordi de er irrelavante for mig.\\
Ronni: Det var det.\\
Luca: Ja.\\
Dennis: Fedt.\\
Luca: Tak skal du have.\\
Dennis: Det var så lidt.\\

\section{Interview med Peter Jensen fra Friluftsland} \label{app:friluftsland}
Her interviewer Daniel og Jens, Peter fra Friluftsland.\\
Daniel: “Jeg sidder her med”\\
Peter: “Peter Jensen\\
Daniel: “Yes”\\
Peter: “Friløbsland” \\
Daniel: “Og øh vi vil godt spørge hvilke system i bruger nu til vagtplanlægning så”\\
Peter: “Vi bruger et system der hedder staffweb øh punktum dk øhm ja”\\
Daniel: “Ja er der sådan umiddelbart nogle problemer i oftest støder på når i bruger det system”\\
Peter: “Nej det køre ret smertefrit” \\
Jens: “Det køre ret smertefrit altså hvordan fungere det, hvad er øh”\\
Peter: “Jamen man kan sige du har øh du har ligesom et hovedprogram som som er det jeg  som jeg sider at planlægger i, hvor jeg kan ligge vagtplanerne så kan jeg samtidig bliver der holdt op imod lønbudgetter, øh ja så jeg kan se hver gang jeg sætter en mand på også kan jeg også se hvad det koster mig, man kan sig hvor alle timelønninger øh"\\
Jens: “Ja”\\
Peter: “Skæv arbejdstid”\\
Jens: “Lige præcis”\\
Peter: “Øhm altså taxerne for det er også lagt der ind, øhm hvad der nu er, så hver gang jeg plotter en time ind, jamen den time afhængig af hvor jeg sætter den så kan jeg som om det er med søndagstillæg jamen så så koster det dem det og det at have den og den mand på arbejdet så jeg hele tiden”\\
Jens: ”Øhm”\\
Peter: “Kan følge med i lønbudgettet samtidig”\\
Jens: “Øhm øhm”\\
Peter: “Ikk øh samtidig kan du nemt øh byt rundt på vagter altså det er lidt"\\
Jens: “Ja”\\
Peter: “Som at sidde at flytte ja på en computeren ikk, sidde at flytte øh mande timerne rundt øh”\\
Jens: “øhm”\\
Peter: “og der kan trækkes forskellige data ud af det os øh timesedler pr mand selvfølgelig øh gennemsnit over året osv. osv.”\\
Jens: “Øhm øhm”\\
Peter: “Ferie hvad der er tilbage af ferie og hvad der er brugt af ferier osv. øhm"\\
Jens: “Yea”\\
Peter: “Også bliver det hele frigivet, øh jeg kan sidde og arbejde i planerne også kan jeg beslutte mig for hvornår vagtplanerne, nu sidder jeg og kigger på sommerferie nu her øhm men alle de rettelser jeg sidder og laver øh er der ingen grund til at medarbejderne kan se før den endelig plan ligger færdig”\\
Jens: “øhm lige præcis”\\
Peter: “Så vælger ligesom en period, jamen så sig nu frigiver jeg den her også integrere system så med folk egen telefoner og computer osv. øhm det almindelig kalender folk bruger g-mail kalender og outlook hvad der nu ellers er”\\
Jens: “Så den er hooket op til alle de forskellig kalender på det forskellig platfromer”\\
Peter: “Ja”\\
Jens: “altså så det øhh okay okay”\\
Peter: “Så så man kan sige har du vagter her ved mig og oprettet i mit system så det at du får en vagt på automatisk inden i din kalender”\\
Jens: “Får de så os ind notification et eller andet der ligesom fortæller dem der er lavet en ændring”\\
Peter: “Øhm nej”\\
Jens: “Eller skal de selv ind og tjekke det”\\
Peter: “De skal selv ind og tjekke det, den synkroniserer bare kalender”\\
Jens: “øhm øhm øhm”\\
Peter: “Jeg mener ikke der kommer noget” \\
Jens: “Nej”\\
Peter: “Den del af det ser jeg sjældent”\\
Jens: “Ja ja det lidt det, det sådan noget der interessant og lige og ikke”\\
Peter: “Yea”\\
Jens: “Også tænkte jeg på øh, hvis nu en f.eks. en øh en øh en medarbejder bliver syg og ikke har muligheden for at tag vagten, kan han så fra sin, fra sin platform sige at jeg syg eller skal han have kontakt til dig som så skal omlægge eller hvordan fungere det”\\
Peter: “Han han skal ringe til mig”\\
Jens: “Okay”\\
Peter: “Det kan han ikke gør”\\
Jens: “Nej”\\
Peter: “Han har kun, den eneste del han har adgang til, både via deres hjemmeside men også, os ja via den her almindelig synkronisering det er endelig bare kalenderen, han kan bare se hvem og hvad der er på”\\
Jens: “Øhm øhm”\\
Peter: “Så han kan se hvem han skal arbejde med osv. men han kan ikke rigtig gøre andet”\\
Jens: “Nej nej”\\
Peter: “Jo han kan ligge en vagt, der er en vagt børs der på øhm men vi synes om det ikke funger ved Mathias”\\
Jens: “Nej nej nej”\\
Peter: “Der der der skal jeg have besked"\\
Jens: “Jaer jaer jaer så alt det er ligesom dig der har, dig der har som der endelig, der er hvad ansvarlig for for for udidikere vagterne også har de sådan set ikke så  meget saying i, det ligsom hvis de skal have byttet vagterne skal det have kontakt igennem dig”\\
Peter: “Ja de kan”\\
Jens: “Okay”\\
Peter: “Også ligge vagterne op i en bytte børs”\\
Jens: “Okay okay”\\
Peter: “Så får jeg besked øh så får jeg besked om når nogen har budt ind på den vagt så kan jeg så godkende det eller afvise det. øh hvis det "\\
Jens: “Jaer jaer så man bytter (?)”\\
Peter: “Af en eller anden årsag ikke eller ja eller en der, der ikke kan åbne og luk f.eks."\\
Jens: “Ja okay ja selvfølgelig jaer”\\
Peter: “(?), hvis det er en øh der er ansvarlig og har nøglerne og han bytter med en øh med den nyeste medarbejder som ikke har åbnet og lukket før, jamen så så (?) hov den går ikke den her”\\
Jens: “Ja jaer”\\
Peter: “Også bliver det ikk øh også bliver den selvfølgelig ikke synkroniseret før den er godkendt vagten”\\
Jens: “Jaer, det vært fald det øh, det øh, det fordel i os altså, at selv skulle øh (?)”\\
Peter: “Ja altså der er en bagpen på det ikk”\\
Jens: “Det er det bestemt øhm”\\
Peter: “Hele systemet køre også med et almendelig øhm hvad hedder det 10 points registreringssystem, så når medarbejderne møder ind her i butikken så logger de ind med deres personale numre også laver den ligesom man før et gammeldags stempelkort”\\
Jens: “Ja lige præcis”\\
Peter: “Det gør den bare elektronisk og de vagter skal igen os øh godkendes, der kommer den og siger hvis det afviger fra den plan der er lagt øh især som folk møder øh kvart i 10 til kvart i 6 i dag så øh du først checker ind kvart over 10 så vil jeg få beskeden om du er mødt for sent kan man sige din timesats eller din timeseddel selvfølgelig bliver rettet derefter, udfra igen en godkendelse fra min side af”\\
Jens: “Yeam uhm uhm, okay når i så skal check ind øh er det så igennem en øh skal i så her på computeren eller har i de en platform det ligesom kan checke ind fra”\\
Peter: “De checker ind på kasse systemet”\\
Jens: “På kasse systemet okay okay, smart nok”\\
Peter: “Ja eller et lille program der køre ved siden af, men ja”\\
Jens: “Ja jaem”\\
Daniel: “Når du så skal til at lægge planerne til at starte med er der så sådan et autocomplet eller skal du selv ind og træk mand ind hverdag”\\
Peter: “Øhm jeg ligger noget skabeloner, man kan sige alle de fuldtidsansatte herinde, jamen de har en, en fast rolle der køre og den, det lagt ind øh"\\
Jens: “Uhm”\\
Peter: “Der laver jeg rollet, siger jamen det skal start fra den dato også ugentligt f.eks."\\ 
Jens: “Jaer jaer ja”\\
Peter: “Også så køre den øhm automatisk den del af det”\\
Daniel: “Ja”\\
Peter: “Også plotter den så sammen, samtidig kan jeg lave en rol for ledig vagter kan man sige, altså hvis der er ind fultidigs mand der altid har fri om onsdagen, jamen så øh så øh kan jeg samtidig, ha have en plan der modsvare den, som der så altid gør at jeg har en ledig vagt om onsdagen som jeg skal have sat”\\
Jens: “Uhm”\\
Daniel: “Ja”\\
Jens: “Og det uhm, du oplever ikke nogen problemer med øh altså øh med vagtplanen der er ikke nogen der ikke øh altså når når nogen skal sættes op og sådan noget der øh skal de så medarbejderne selv ind og sætte det hele op eller øh eller øh viser i hvordan man gøre eller hvordan funger det”\\
Peter: “Det det bliv gjort altså de får tildelt vagterne”\\
Jens: “Uhm”\\
Peter: “Også øhm så, altså du tænker sæt op på telefonen eller øh”\\
Jens: “Hele hele sæt oppet med man får det ind på sin outlook og man får det ind på sin øh google kalender, altså øh”\\
Peter: “Der er en vejledning når man øhm altså det har en, en hjemmeside adgang igen med deres øh personale numre også videre og fra den ligger der bare en vejledning hvis man vil synkroniser det, med sin kalender også sin almindelig kalender"\\
Jens: “Jaer”
Peter: “Så ligger der vejledninger der til det"\\
Jens: “Okay”\\
Peter: “Som, så den vej rundt kan de følge det”\\
Jens : “Uhm uhm”\\
Peter: “Og ellers så, er har jeg og har også hjulpet nogen med at sætte det op”\\
Jens: “Jaja selvfølgelig”\\
Peter: “Men det går rimeligt smertefrit”\\
Jens: “Det også, altså, udmærket udmærket, men ellers der ikke nogen problemer med øh, der er ikke nogen, systemet ikke fungerer eller det offline eller\\
Peter: “Nej“\\
Jens: “Det er der sgu ikk”\\
Peter: “Nej det køre pokkers pokkers”\\
Jens: “Ja, ja ej det smukt”\\
Peter: “Pokkers let”\\
Jens: “Det smukt”\\
Peter: “Det gør det, men hvis der skulle være noget så skulle det være, kan man sige den her øh styre enhederne eller øhm hva skulle man sige, det program jeg sidder med”\\
Jens: “Uhm”\\
Peter: “Det er ikke mobil venligt, altså det kan jeg ikke lig på en, en tablet”\\
Jens: “Uhm nej nej”\\
Peter: “Ellers en telefon”\\
Jens: “Altså nej”\\
Peter: “F.eks.”\\
Jens: “ Så hvis der er nogen der”\\
Peter: “ Så man kunne sige hvis jeg skulle ændre i det, så skal jeg sidder her, altså jeg skal kunne sidde på en computer, hvor det er installeret, jeg kan ikke gøre det f.eks. via en, en webadgang, så skal jeg ha et almindeligt fjerne skrivebord til min arbejdscomputer også den vej rundt, men men øh det ikke særlig hænsigmæsigt via, altså kun en mobilapplikation måske være en mulighed”\\
Jens: “Ja det er det”\\
Peter: “Hvis det var sådan at man skulle have noget at bygge over”\\
Jens: “Ja det er det der med igen hvor flexibel det hvis der lige pludseligt er nogen der ringer ind jeg er syg der, så så skal man hele vejen over til computeren ikk, også skal man sidde og øh skribler der og øh”\\
Peter: “Og det er forudsat at jeg er her”\\
Jens: “Lige præcis, lige præcis”\\
Peter: “Hvis jeg så sidder et andet sted til mød eller med dem, så har jeg ikke mulighed for at ændre i det, eller øh systemet kan også sende sms’er ud igen til, til personalet hvis man har en vagt man, man en sygemelding, der kommer ind, jamen så kan man jo bare marker vagten også sig øhm send til alle dem der i forvejen ikke er på vagtplannen, så for de en sms, hvor det igen har mulighed for at og øh, og svar reture den kan de godt tage eller det kan være der kl 12:00 eller hvad ved jeg”\\
Jens: “ja ja mhh”\\
Peter: “Så øhm men så skal jeg igen ind og øh og have fat i computeren øhm”\\
Jens: “Ja”\\
Peter: “For at køre det"\\
Jens: “Uhm”\\
Daniel: “Jaem”\\
Peter: “Så øhm det kun være et forbedringspunkt til øhm”\\
Jens: “Jaem"\\
Peter: “Til det system”\\
Jens: “Uhm ja”\\
Daniel: “Udover det så er der ikke nogen bestemte ting du tænker det kunne være fedt hvis det endelig kun et eller andet”\\
Peter: “Nej jeg er bang for, det kan mere ind jeg bruger”\\
Jens: “Ja hehehe”\\
Peter: “Det har flere knapper ind jeg trykker på øhm"\\
Jens: “Uhm uhm uhm”\\
Peter: “Nej fordi det er de der forskellig, altså ehm man kan sig noget af det som øhm, som skal være med det er jo de der forskellig øhm, det er jo det der forskellig rapport udtrækningere, man kan lave på selvfølelig pr mand for at lave, lave løn ikk, men øhm men også øhm at du kan sidde når du gør året op eller øhm halvåret eller hvad man nu gør, så sidde og sige hvor meget har de forskellige afløser, f.eks. hvor meget har det endelig været her, øhm eller bytter de netop alle deres vagter væk også, så det nemt og træk sådan nogen  ting ud øhm"\\
Jens: “Uhm”\\
Peter: “Det øhm kan man så sige, det er flere mennesker man har tilknyttets, desto mere er det nødvendigt, fordi så, så kan man jo sige en store virksomhed, så mister man øhm, måske lidt overblikket over “\\
Jens: “Ja altså”\\
Peter: “Flaskedrengene der for putter sig eller hvad eller hva kan det være”\\
Jens: “Præcis lige præcis”\\
Peter: “Så øhm”\\
Jens: “Det skal enten være automatisk eller så skal det være ligesom være underdelinger ikke”\\
Peter: “Ja”\\
Jens: “Jaem”\\
Peter: “Præcis”\\
Jens: “Jaem”\\
Daniel: “Jaem”\\
Jens: “Men det fair nok”\\
Daniel: “Ja”\\
Jens: “Har vi øh det var cool”\\
Daniel: “Ja det som”\\
Jens: “Lyder som i har et udmærket system”\\
Peter: “Ja, det øh det fungere i hvert fald”\\
Jens: “Ja det, det er lækkert”\\
Peter: “u på klageligt”\\
Jens: “Uhm”\\
Peter: “Så det øhm men det jog igen tid tidens tand layoutet af det er meget, (?)"\\ 
Jens: “Ja hihi”\\
Peter: “Programmeret eller hvad man nu skal sige, det det det kun man godt se, men altså”\\
Jens: “Ja ja jo det jo ikk”\\
Peter: “Der måske noget brugervenlighed man måske kunne arbejde med"\\
Daniel: “Ja”\\
Jens: “Uhm”\\
Peter: “Hvor det ikke er så intuitivt”\\
Jens: “Uhm ja"\\
Daniel: “Jamen så vil vi sige”\\
Peter: “Det jo nok nå vi alle går rundt med sådan en der i lommen så synes vi alt andet er ret bøvlet ikk”\\
Jens: “Ja ja det rigtig, der er sat en ny standard”\\
Peter: “Ja”\\
Jens “Jamen det er fair nok”\\
Daniel: “Så vil vi sige tak”\\
Jens: “Det var lækkert”\\